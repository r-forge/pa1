\title{Performance Attribution for Equity Portfolios}
\author{by Yang Lu and David Kane}

\maketitle

\setkeys{Gin}{width=0.95\textwidth}

\abstract{

The \pkg{pa} package provides tools for conducting performance
attribution for equity portfolios. The package uses two methods: the
Brinson-Hood-Beebower model (hereafter referred to as the Brinson
model) and a regression-based analysis. The Brinson model takes an
ANOVA-type approach and decomposes the active return of any portfolio
into asset allocation, stock selection, and interaction effect. The
regression-based analysis utilizes estimated coefficients, based on a
regression model, to attribute active return to different
factors. 
}
\section{Introduction}

Almost all portfolio managers measure performance with reference to a
benchmark. The difference in return between a portfolio and the
benchmark is its active return. Performance attribution decomposes the
active return. The two most common approaches are the Brinson model
and a regression-based analysis.\footnote{See \citet{brinson:gary}
and \citet{jpmreg} for more information.}


The \pkg{pa} package provides tools for conducting both methods for
equity portfolios. The Brinson model takes an ANOVA-type approach and
decomposes the active return of any portfolio into asset allocation,
stock selection, and interaction effects. The regression-based
analysis utilizes estimated coefficients from a linear model to
estimate the contributions from different factors. 
\section{Data}

We demonstrate the use of the \pkg{pa} package with a series of
examples based on data from MSCI Barra's Global Equity Model II
(GEM2).\footnote{See www.msci.com and \citet{gem2} for more
  information.} The original data set contains selected attributes
such as industry, size, country, and various style factors for a
universe of approximately 48,000 securities on a monthly basis. For
illustrative purposes, this article uses three modified versions of
the original data set (\texttt{year}, \texttt{quarter}, and
\texttt{jan}), each containing 3000 securities.  The data frame,
\texttt{quarter}, is a subset of \texttt{year}, containing the data of
the first quarter. The data frame, \texttt{jan}, is a subset of
\texttt{quarter} with the data from January, 2010.
\begin{smallverbatim}
> data(year)
> names(year)
 [1] "barrid"    "name"      "return"   
 [4] "date"      "sector"    "momentum" 
 [7] "value"     "size"      "growth"   
[10] "cap.usd"   "yield"     "country"  
[13] "currency"  "portfolio" "benchmark"
\end{smallverbatim}

See \texttt{?year} for information on the different variables. 
The top 200 securities, based on \texttt{value} scores,
  in January are selected as portfolio holdings and are held through
  December 2010 with monthly rebalances to maintain equal-weighting.
  The benchmark for this portfolio is defined as the largest 1000 
  securities based on \texttt{size} each month. The
  benchmark is cap-weighted.

Here is a sample of rows and columns from the data frame
\texttt{year}:
\begin{smallverbatim}
                                      name
44557  BLUE STAR OPPORTUNITIES CORP       
25345  SEADRILL                           
264017 BUXLY PAINTS (PKR10)               
380927 CDN IMPERIAL BK OF COMMERCE        
388340 CDN IMPERIAL BK OF COMMERCE        
         return       date     sector  size
44557   0.00000 2010-01-01     Energy  0.00
25345  -0.07905 2010-01-01     Energy -0.27
264017 -0.01754 2010-05-01  Materials  0.00
380927  0.02613 2010-08-01 Financials  0.52
388340 -0.00079 2010-11-01 Financials  0.55
       country portfolio benchmark
44557      USA     0.000  0.000000
25345      NOR     0.000  0.000427
264017     PAK     0.005  0.000000
380927     CAN     0.005  0.000012
388340     CAN     0.005  0.000012
\end{smallverbatim}

The portfolio has 200 equal-weighted holdings each month. The row 
for Canadian Imperial Bank of Commerce indicates that it is one of the 200
portfolio holdings with a weight of 0.5\% in 2010. Its return was
2.61\% in August, and close to flat in November.
\section{Brinson model}

Consider an equity portfolio manager who uses the S\&P 500 as the
benchmark. In a given month, she outperformed the S\&P by 3\%. Part of
that performance was due to the fact that she allocated more weight of
the portfolio to certain sectors that performed well. Call this the
\emph{allocation effect}. Part of her outperformance was due to the
fact that some of the stocks she selected did better than their sector
as a whole. Call this the \emph{selection effect}. The residual can
then be attributed to an interaction between allocation and selection
-- the \emph{interaction effect}. The Brinson model provides
mathematical definitions for these terms and methods for calculating
them.

The example above uses sector as the classification scheme when
calculating the allocation effect. But the same approach can work with
any other variable which places each security into one, and only one,
discrete category: country, industry, and so on. In fact, a similar
approach can work with continuous variables that are split into
discrete ranges: the highest quintile of market cap, the second
highest quintile and so forth. For generality, we will use the term
``category'' to describe any classification scheme which places each
security in one, and only one, category.

Notations:
\begin{itemize}
\item $w^B_i$ is the weight of security $i$ in the benchmark.
\item $w^P_i$ is the weight of security $i$ in the portfolio.
\item $W^B_j$ is the weight of category $j$ in the benchmark. $W^B_j
  = \sum w^B_i$, $i$ $\in$ $j$.
\item $W^P_j$ is the weight of a category $j$ in the portfolio. $W^P_j
  = \sum w^P_i$, $i$ $\in$ $j$.
\item The sum of the weight $w^B_i$, $w^P_i$, $W^B_j$, and $W^P_j$ is 1, respectively.
  \item $r_i$ is the return of security $i$.
\item $R^B_j$ is the return of a category $j$ in the benchmark. $R^B_j
  = \sum w^B_ir_i$, $i$ $\in$ $j$.
\item $R^P_j$ is the return of a category $j$ in the portfolio. $R^P_j
  = \sum w^P_ir_i$, $i$ $\in$ $j$.
\end{itemize}

The return of a portfolio, $R_P$, can be calculated in two ways:
\begin{itemize}
  \item On an individual security level by summing over $n$ stocks: $R_P =
    \sum\limits_{i = 1}^n w^P_ir_i$.
  \item On a category level by summing over $N$ categories: $R_P = \sum\limits_{j =
    1}^N W^P_jR^P_j$.
\end{itemize}

Similar definitions apply to the return of the benchmark, $R_B$,
\begin{itemize}
  \item $R_B = \sum\limits_{i = 1}^n w^B_ir_i$.
  \item $R_B = \sum\limits_{j = 1}^N W^B_jR^B_j$.
\end{itemize}

Active return of a portfolio, $R_{active}$, is a performance measure
of a portfolio relative to its benchmark. The two conventional
measures of active return are arithmetic and geometric.  The \pkg{pa}
package implements the arithmetic measure of the active return for a
single-period Brinson model because an arithmetic difference is more
intuitive than a ratio over a single period.

The arithmetic active return of a portfolio, $R_{active}$, is the
portfolio return $R_P$ less the benchmark return $R_B$:
\begin{center}
  $R_{active} = R_P - R_B$.
\end{center}

Since the category weights of the portfolio are generally different
from those of the benchmark, allocation plays a role in the active
return, $R_{active}$. The same applies to stock selection effects. 
Within a given category, the portfolio and the benchmark will rarely 
have exactly the same holdings. Allocation effect
$R_{allocation}$ and selection effect $R_{selection}$ over $N$
categories are defined as:
\begin{center}
  $R_{allocation} = \sum\limits_{j = 1}^N W^P_jR^B_j - \sum\limits_{j
    = 1}^N W^B_jR^B_j$,
  $R_{selection} = \sum\limits_{j = 1}^N W^B_jR^P_j - \sum\limits_{j =
    1}^N W^B_jR^B_j$.
\end{center}

The intuition behind the allocation effect is that a portfolio would
produce different returns with different allocation schemes ($W^P_j$
vs. $W^B_j$) while having the same stock selection and thus the same
return ($R^B_j$) for each category. The difference between the two
returns, caused by the allocation scheme, is called the allocation
effect ($R_{allocation}$). Similarly, two different returns can be
produced when two portfolios have the same allocation ($W^B_j$) yet
dissimilar returns due to differences in stock selection within each
category ($R^p_j$ vs. $R^B_j$). This difference is the selection
effect ($R_{selection}$).

Interaction effect ($R_{interaction}$) is the result of subtracting
return due to allocation $R_{allocation}$ and return due to selection
$R_{selection}$ from the active return $R_{active}$:
\begin{center}
  $R_{interaction} = R_{active} - R_{allocation} - R_{selection}$.
\end{center}

The Brinson model allows portfolio managers to analyze the active
return of a portfolio using any attribute of a security, such as
country or sector. Unfortunately, it is very hard to expand the
analysis beyond two categories. As the number of categories increases,
this procedure is subject to the curse of dimensionality. To some
extent, the regression-based model detailed later ameliorates this
problem.
\subsection{Brinson tools}

Brinson analysis is run by calling the function \texttt{brinson} to
produce an object of class \texttt{brinson}. 
\begin{smallverbatim}
> data(jan)
> br.single <- brinson(x = jan, date.var = "date", 
+            cat.var = "sector",
+            bench.weight = "benchmark", 
+            portfolio.weight = "portfolio", 
+            ret.var = "return")
\end{smallverbatim}

The data frame, \texttt{jan}, contains all the information necessary
to conduct a single-period Brinson analysis. \texttt{date.var},
\texttt{cat.var}, and \texttt{return} identify the columns containing
the date, the factor to be analyzed, and the return variable,
respectively. \texttt{bench.weight} and \texttt{portfolio.weight}
specify the name of the benchmark weight column and that of the
portfolio weight column in the data frame.

Calling \texttt{summary} on the resulting object \texttt{br.single} of
class \texttt{brinson} reports essential information about the input
portfolio (including the number of securities in the portfolio and the
benchmark as well as sector exposures) and the results of the Brinson
analysis.

\begin{smallverbatim}
> summary(br.single)
Period:                            2010-01-01
Methodology:                       Brinson
Securities in the portfolio:       200
Securities in the benchmark:       1000

Exposures 
            Portfolio Benchmark
Energy          0.085    0.2782
Materials       0.070    0.0277
Industrials     0.045    0.0330
ConDiscre       0.050    0.0188
ConStaples      0.030    0.0148
HealthCare      0.015    0.0608
Financials      0.370    0.2979
InfoTech        0.005    0.0129
TeleSvcs        0.300    0.1921
Utilities       0.030    0.0640

Returns 
                   2010-01-01
Allocation Effect    -0.00140
Selection Effect      0.01418
Interaction Effect    0.00191
Active Return         0.01469
\end{smallverbatim}

The \texttt{br.single} summary shows that the active return of the
portfolio, in January, 2010 was 1.47\%. This return can be decomposed
into allocation effect (-0.14\%), selection effect (1.42\%), and
interaction effect (0.19\%).
\begin{smallverbatim}
> plot(br.single, var = "sector", type = "return")
\end{smallverbatim}
\begin{figure}
\centering
\vspace*{.1in}
\includegraphics{LuKane-006}
\caption{\label{figure:return}
  Sector Return.}
\end{figure}

Figure \ref{figure:return} is a visual representation of the return of
both the portfolio and the benchmark sector by sector in January,
2010. Utilities was the
sector with the highest active return in the portfolio.

To obtain Brinson attribution on a multi-period data set, one
calculates allocation, selection and interaction within each period
and aggregates them across time. There are three methods for this --
arithmetic, geometric, and optimized linking. The arithmetic attribution model
calculates active return and contributions due to allocation,
selection, and interaction in each period and sums them over multiple
periods. 

In practice, analyzing a single-period portfolio is meaningless as
portfolio managers and their clients are more interested in the
performance of a portfolio over multiple periods. To apply the Brinson
model over time, we can use the function \texttt{brinson} and input a
multi-period data set (for instance, \texttt{quarter}) as shown
below. 

\begin{smallverbatim}
> data(quarter)
> br.multi <- brinson(quarter,
+           date.var = "date",
+           cat.var = "sector",
+           bench.weight = "benchmark",
+           portfolio.weight = "portfolio",
+           ret.var = "return")
\end{smallverbatim}
The object \texttt{br.multi} of class \texttt{brinsonMulti} is an
example of a multi-period Brinson analysis.
\begin{smallverbatim}
> exposure(br.multi, var = "size")
$Portfolio
     2010-01-01 2010-02-01 2010-03-01
Low       0.140      0.140      0.155
2         0.050      0.070      0.045
3         0.175      0.145      0.155
4         0.235      0.245      0.240
High      0.400      0.400      0.405

$Benchmark
     2010-01-01 2010-02-01 2010-03-01
Low      0.0681     0.0568     0.0628
2        0.0122     0.0225     0.0170
3        0.1260     0.1375     0.1140
4        0.2520     0.2457     0.2506
High     0.5417     0.5374     0.5557
\end{smallverbatim}

The \texttt{exposure} method on the class \texttt{br.multi} object
shows the exposure of the portfolio and the benchmark based on a
user-specified variable. Here, it shows the exposure on
\texttt{size}. We can see that the portfolio overweights the benchmark
in the lowest quintile in \texttt{size} and underweights in the
highest quintile.
\begin{smallverbatim}
> returns(br.multi, type = "arithmetic")
$Raw
             2010-01-01 2010-02-01 2010-03-01
Allocation      -0.0014     0.0062     0.0047
Selection        0.0142     0.0173    -0.0154
Interaction      0.0019    -0.0072    -0.0089
Active Return    0.0147     0.0163    -0.0196

$Aggregate
             2010-01-01, 2010-03-01
Allocation                   0.0095
Selection                    0.0160
Interaction                 -0.0142
Active Return                0.0114
\end{smallverbatim}

The \texttt{returns} method shows the results of the Brinson analysis
applied to the data from January, 2010 through March, 2010. The first 
portion of the \texttt{returns} output shows
the Brinson attribution in individual periods. The second portion
shows the aggregate attribution results. The portfolio formed by top
200 value securities in January had an active return of 1.14\% over
the first quarter of 2010. The allocation and the selection effects
contributed 0.95\% and 1.6\% respectively; the interaction effect
decreased returns by 1.42\%.
\section{Regression}

One advantage of a regression-based approach is that such analysis
allows one to define their own attribution model by easily
incorporating multiple variables in the regression formula. These
variables can be either discrete or continuous.

Suppose a portfolio manager wants to find out how much each of the
value, growth, and momentum scores of her holdings contributes to the
overall performance of the portfolio. Consider the following linear
regression without the intercept term based on a single-period
portfolio of $n$ securities with $k$ different variables:
\begin{center}
  $\mathbf{r}_n = \mathbf{X}_{n,k}\mathbf{f}_k + \mathbf{u}_n$
\end{center}
where 
\begin{itemize}
\item $\mathbf{r}_n$ is a column vector of length $n$. Each element in
  $\mathbf{r}_n$ represents the return of a security in the portfolio.
\item $\mathbf{X}_{n,k}$ is an $n$ by $k$ matrix. Each row represents
  $k$ attributes of a security. There are $n$ securities in the
  portfolio.
\item $\mathbf{f}_k$ is a column vector of length $k$. The elements
  are the estimated coefficients from the regression. Each element
  represents the \emph{factor return} of an attribute.
\item $\mathbf{u}_n$ is a column vector of length $n$ with residuals
  from the regression.
\end{itemize}

In the case of this portfolio manager, suppose that she only has three
holdings in her portfolio. $r_3$ is thus a 3 by 1 matrix with returns
of all her three holdings. The matrix $\mathbf{X}_{3,3}$ records the
score for each of the three factors (value, growth, and momentum) in
each row. $\mathbf{f}_3$ contains the estimated coefficients of a
regression $\mathbf{r}_3$ on $\mathbf{X}_{3, 3}$.

The active exposure of each of the $k$ variables, $X_{i}$, $i$
$\in$ $k$, is expressed as
\begin{center}
  $X_i = \mathbf{w}_{active}\prime \mathbf{x}_{n,i}$,
\end{center}
where $X_i$ is the value representing the active exposure of the
attribute $i$ in the portfolio, $\mathbf{w}_{active}$ is a column
vector of length $n$ containing the active weight of every security in
the portfolio, and $\mathbf{x}_{n, i}$ is a column vector of length
$n$ with attribute $i$ for all securities in the portfolio. Active
weight of a security is defined as the difference between the
portfolio weight of the security and its benchmark weight.

Using the example mentioned above, the active exposure of the
attribute \texttt{value}, $X_{value}$ is the product of
$\mathbf{w}_{active}\prime$ (containing active weight of each of the
three holdings) and $\mathbf{x}_{3}$ (containing value scores of the
three holdings).

The contribution of a variable $i$, $R_i$, is thus the product of the
factor returns for the variable $i$, $f_i$ and the active exposure of
the variable $i$, $X_i$. That is,
\begin{center}
  $R_i = f_iX_i$.
\end{center}
Continuing the example, the contribution of value is the product of
$f_{value}$ (the estimated coefficient for value from the linear
regression) and $X_{value}$ (the active exposure of value as shown
above).

Therefore, the active return of the portfolio $R_{active}$ is the sum
of contributions of all $k$ variables and the residual $u$
(a.k.a. the interaction effect),
\begin{center}
  $R_{active} = \sum\limits_{i = 1}^kR_i + u$.
\end{center}

For instance, a hypothetical portfolio has three holdings (A, B, and
C), each of which has two attributes -- size and value.

\begin{smallverbatim}
  Return Name Size Value Active_Weight
1    0.3    A  1.2   3.0           0.5
2    0.4    B  2.0   2.0           0.1
3    0.5    C  0.8   1.5          -0.6
\end{smallverbatim}

Following the procedure as mentioned, the factor returns for size and
value are -0.0313 and -0.1250. The active exposure of size is 0.32
and that of value is 0.80. The active return of the portfolio is -11\%
which can be decomposed into the contribution of size and that of
value based on the regression model. Size contributes 1\% of the
negative active return of the portfolio and value causes the portfolio
to lose the other 10.0\%.
\subsection{Regression tools}

The \pkg{pa} package provides tools to analyze
both single-period and multi-period data frames.
\begin{smallverbatim}
> rb.single <- regress(jan, date.var = "date",
+          ret.var = "return",
+          reg.var = c("sector", "growth", 
+                      "size"),
+          benchmark.weight = "benchmark",
+          portfolio.weight = "portfolio")
> exposure(rb.single, var = "growth")
     Portfolio Benchmark
Low      0.305    0.2032
2        0.395    0.4225
3        0.095    0.1297
4        0.075    0.1664
High     0.130    0.0783
\end{smallverbatim}

\texttt{reg.var} specifies the columns containing variables whose
contributions are to be analyzed. Each of the \texttt{reg.var} input
variables corresponds to a particular column in $\mathbf{X}_{n,k}$
from the aforementioned regression model. \texttt{ret.var} specifies
the column in the data frame \texttt{jan} based on which
$\mathbf{r}_n$ in the regression model is formed.
\begin{smallverbatim}
> exposure(rb.single, var = "growth")
     Portfolio Benchmark
Low      0.305    0.2032
2        0.395    0.4225
3        0.095    0.1297
4        0.075    0.1664
High     0.130    0.0783
\end{smallverbatim}

Calling \texttt{exposure} with a specified \texttt{var} yields
information on the exposure of both the portfolio and the benchmark by
that variable. If \texttt{var} is a continuous variable, for instance,
\texttt{growth}, the exposure will be shown in 5 quantiles. Majority
of the high \texttt{value} securities in the portfolio in January have
relatively low \texttt{growth} scores.
\begin{smallverbatim}
> summary(rb.single)
Period:                            2010-01-01
Methodology:                       Regression
Securities in the portfolio:       200
Securities in the benchmark:       1000

Returns 
                 2010-01-01
sector             0.003189
growth             0.000504
size               0.002905
Residual           0.008092
Portfolio Return  -0.029064
Benchmark Return  -0.043753
Active Return      0.014689
\end{smallverbatim}

The \texttt{summary} method shows the number of securities in the
portfolio and the benchmark, and the contribution of each input
variable according to the regression-based analysis. In this case, the
portfolio made a loss of 2.91\% and the benchmark lost
4.38\%. Therefore, the portfolio outperformed the benchmark by
1.47\%. \texttt{sector}, \texttt{growth}, and \texttt{size}
contributed 0.32\%, 0.05\%, and 0.29\%, respectively.

Regression-based analysis can be applied to a multi-period data frame
by calling the same method \texttt{regress}. By typing the name of the
class object \texttt{rb.multi} directly, a short summary of the
analysis is provided, showing the starting and ending period of the
analysis, the methodology, and the average number of securities in
both the portfolio and the benchmark.
\begin{smallverbatim}
> rb.multi <- regress(year, date.var = "date",
+         ret.var = "return",
+         reg.var = c("sector", "growth", 
+                     "size"),
+         benchmark.weight = "benchmark",
+         portfolio.weight = "portfolio")
> rb.multi
Period starts:                     2010-01-01
Period ends:                       2010-12-01
Methodology:                       Regression
Securities in the portfolio:       200
Securities in the benchmark:       1000
\end{smallverbatim}

The regression-based summary shows that the contribution of each input
variable in addition to the basic information on the portfolio. The
summary suggests that the active return of the portfolio in year 2010
is 10.1\%. The \texttt{Residual} number indicates the contribution
of the interaction among various variables including \texttt{sector},
\texttt{growth}, and \texttt{growth}. Based on the
regression model, \texttt{size} contributed to the lion share of the
active return.
\begin{smallverbatim}
> summary(rb.multi)
Period starts:                     2010-01-01
Period ends:                       2010-12-01
Methodology:                       Regression
Avg securities in the portfolio:   200
Avg securities in the benchmark:   1000

Returns 
$Raw
                 2010-01-01 2010-02-01 2010-03-01
sector               0.0032     0.0031     0.0002
growth               0.0005     0.0009    -0.0001
size                 0.0029     0.0295     0.0105
Residual             0.0081    -0.0172    -0.0302
Portfolio Return    -0.0291     0.0192     0.0298
Benchmark Return    -0.0438     0.0029     0.0494
Active Return        0.0147     0.0163    -0.0196
                 2010-04-01 2010-05-01 2010-06-01
sector               0.0016     0.0039     0.0070
growth               0.0001     0.0002     0.0004
size                 0.0135     0.0037     0.0018
Residual            -0.0040     0.0310     0.0183
Portfolio Return    -0.0080    -0.0381     0.0010
Benchmark Return    -0.0192    -0.0769    -0.0266
Active Return        0.0113     0.0388     0.0276
                 2010-07-01 2010-08-01 2010-09-01
sector               0.0016     0.0047    -0.0022
growth              -0.0005     0.0005    -0.0006
size                 0.0064     0.0000     0.0096
Residual            -0.0324     0.0173    -0.0220
Portfolio Return     0.0515    -0.0119     0.0393
Benchmark Return     0.0764    -0.0344     0.0545
Active Return       -0.0249     0.0225    -0.0152
                 2010-10-01 2010-11-01 2010-12-01
sector               0.0015    -0.0044    -0.0082
growth              -0.0010    -0.0004     0.0010
size                 0.0022     0.0130     0.0056
Residual             0.0137     0.0175    -0.0247
Portfolio Return     0.0414    -0.0036     0.0260
Benchmark Return     0.0249    -0.0293     0.0523
Active Return        0.0165     0.0257    -0.0263

$Aggregate
                 2010-01-01, 2010-12-01
sector                           0.0120
growth                           0.0011
size                             0.1030
Residual                        -0.0269
Portfolio Return                 0.1191
Benchmark Return                 0.0176
Active Return                    0.1015
\end{smallverbatim}

Figure \ref{figure:regmultiattrib} displays both the cumulative
portfolio and benchmark returns from January, 2010 through December,
2010. It suggests that the portfolio, consisted of high \texttt{value}
securities in January, consistently outperformed the benchmark in
2010. Outperformance in May and June helped the overall positive
active return in 2010 to a large extent.
\begin{smallverbatim}
> plot(rb.multi, var = "sector", type = "return")
\end{smallverbatim}
\begin{figure}
\centering
\vspace*{.1in}
\includegraphics{LuKane-016}
\caption{\label{figure:regmultiattrib}
  Performance Attribution.}
\end{figure}
\section{Conclusion}

In this paper, we describe two widely-used methods for performance
attribution -- the Brinson model and the regression-based approach,
and provide a simple collection of tools to implement these two
methods in \texttt{R} with the \pkg{pa} package.  A
comprehensive package, \pkg{portfolio} \citet{kane:david}, provides
facilities to calculate exposures and returns for equity
portfolios. It is possible to use the \pkg{pa} package based on the
output from the \pkg{portfolio} package. Further, the flexibility
of \texttt{R} itself allows users to extend and modify these packages
to suit their own needs and/or execute their preferred attribution
methodology. Before reaching that level of complexity, however,
\pkg{pa} provides a good starting point for basic performance
attribution.

%%\bibliography{pa}

\begin{thebibliography}{4}
\providecommand{\natexlab}[1]{#1}
\providecommand{\url}[1]{\texttt{#1}}
\expandafter\ifx\csname urlstyle\endcsname\relax
  \providecommand{\doi}[1]{doi: #1}\else
  \providecommand{\doi}{doi: \begingroup \urlstyle{rm}\Url}\fi

\bibitem[Brinson et~al.(1986)Brinson, Hood, and Beebower]{brinson:gary}
G.~Brinson, R.~Hood, and G.~Beebower.
\newblock {Determinants of Portfolio Performance}.
\newblock \emph{Financial Analysts Journal}, 42\penalty0 (4):\penalty0 39--44,
  Jul.--Aug. 1986.
\newblock URL \url{http://www.jstor.org/stable/4478947}.

\bibitem[Enos and Kane(2006)]{kane:david}
J.~Enos and D.~Kane.
\newblock {Analysing Equity Portfolios in R}.
\newblock \emph{R News}, 6\penalty0 (2):\penalty0 13--19, MAY 2006.
\newblock URL \url{http://CRAN.R-project.org/doc/Rnews}.

\bibitem[Grinold(2006)]{jpmreg}
R.~Grinold.
\newblock {Attribution, Modeling asset characteristics as portfolios}.
\newblock \emph{The Journal of Portfolio Management}, page~19, Winter 2006.

\bibitem[Menchero et~al.(2008)Menchero, Morozov, and Shepard]{gem2}
J.~Menchero, A.~Morozov, and P.~Shepard.
\newblock {The Barra Global Equity Model (GEM2)}.
\newblock \emph{MSCI Barra Research Notes}, Sept. 2008.

\end{thebibliography}

\address{Yang Lu\\
  Williams College\\
  2896 Paresky Center \\
  Williamstown, MA 01267 \\
  United States}\\
\email{yang.lu@williams.edu}

\address{David Kane\\ 
  Harvard University\\ 
  IQSS \\
  1737 Cambridge Street \\
  CGIS Knafel Building, Room 350 \\
  Cambridge, MA 02138\\ 
  United States}\\ 
\email{dave.kane@gmail.com}
